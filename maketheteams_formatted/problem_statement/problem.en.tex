\problemname{Make the teams}

\illustration{0.3}{image.png}{}
% Source: URL to image.

% optionally define variables/limits for this problem
\newcommand{\maxa}{123456789}

In the past, when Gandalf was fighting Darth Vader, there was once a GAPC Edition 0, which consisted of 3 problems: 
one about dynamic programming, one about graphs and one about number theory. 
There were teams consisting of 1 or 2 students. 
There were also some very advanced machine learning algorithms which could precisely assess how difficult each problem was,
and assign it a difficulty value. Moreover, they could also assess how capable a student was at solving a 
certain kind of a problem, and assigning them a corresponding knowledge score, 
by using the precise power of the random number generator. 
In order to solve a problem, a student must have a capability score for that kind of problem that 
is at least equal to the difficulty of the problem. 
It is your job to tell us if it was possible to organise the students in teams of 1 or 2 students, in such a way that all the teams could have solved at least 2 problems in that contest. 
The problems that can be solved have to be different, i.e. if 2 students from the same team can both solve one of the problems,
that does not count as two problems solved.

\begin{Input}
    The input consists of:
    \begin{itemize}
        \item First line consists of three numbers $a, b, c$ with $0 \leq a, b, c \leq 10^{10}$ which represent the difficulty levels of the dynamic programming problem, the graphs problem and the number theory problem, respectively. 
        \item Second line represents the number of students $N$ with $1 \leq N \leq 10^6$ that participated in the contest.
        \item Then, $N$ lines, each containing 3 numbers $x, y, z$ with $0 \leq x, y, z \leq 10^{10}$ which represent a student's knowledge score for the dynamic programming problem, the graphs problem and the number theory problem, respectively.
    \end{itemize}
\end{Input}

\begin{Output}
    'Yes', if it is possible to organise the students in teams of 1 or 2, such that every team solves at least two problems in the contest, otherwise 'No'.
\end{Output}
