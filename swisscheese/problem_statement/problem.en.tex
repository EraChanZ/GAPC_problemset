\problemname{\problemyamlname}

\illustration{0.2}{image.jpeg}{}
% Source: URL to image.

% optionally define variables/limits for this problem
\newcommand{\maxn}{10000}

You were very lucky, and for this summer you got admitted to an internship in a huge software engineering company in switzerland called "Kässoft". 
However your first assignment seemed a bit weird. Recentely a company recieved a task from a local factory making swiss cheese. They decided to enter a new "elite" market by dividing their produced cheese slices in two price categories, "perfect", and "normal".
"Perfection" of the cheese piece is estimated by "roundness" of the holes on a slice. Your task is to write a program that will classify the slice.
For simplicity it was assumed that cheese is a rectangle consisting of "0" and "1", where "0" represent holes, and "1" represent cheese. 
A hole is considered perfect if given its center coordinate (a, b), and integer radius "r", all cells (x, y) with $(x - a)^2 + (y - b)^2 \leq r^2$ are all "0". It is guaranteed, that no hole interesect, or touch each other.
Print classification of the given slice "perfect", or "normal", and if "perfect" also output all holes paramaters, its center coordinate, and a radius.

Note:
\begin{itemize}
    \item Top-left cell of the grid has coordinate (0, 0).
    \item If multiple hole configurations are possible, print the one with the smallest radius sum.
\end{itemize}

\begin{Input}
    The input consists of:
    \begin{itemize}
        \item One line containing two integers $n, m$ ($0 < n\leq \maxn, 0 < m \leq \maxn$), width and height of the slice
        \item $r$, minimum radius of a proper hole
        \item $m$ lines containing $n$ integers "0" or "1" separated by space
    \end{itemize}
\end{Input}

\begin{Output}
    \begin{itemize}
        \item "Perfect", or "normal"
        \item $t$, where $t$ is amount of holes on a slice.
        \item next $t$ lines containing three numbers separated by a space: $x, y, r$, where (x, y) is coordinate of the centher of the hole, and r is its radius.
    \end{itemize}
    \Sample{swisscheese/data/sample/1.in}{swisscheese/data/sample/1.ans}
\end{Output}


